\documentclass[paper=letter,fontsize=10pt]{article}
\usepackage{newcent}
\usepackage{lineno}
\usepackage{hyperref}
\usepackage{enumitem}
\usepackage{xcolor}
\title{\textsc{\textit{Carpenter v. United States} Summary}}
\date{}
\author{\textit{Varun Iyer} \hspace{.5em} $\cdot$ \hspace{.5em} \textit{August 15, 2019}}
\definecolor{maroon}{rgb}{.5,0,0}
\definecolor{navy}{rgb}{0,0,.5}
\newcommand{\pet}[1]{{\color{navy}{#1}}}
\newcommand{\res}[1]{{\color{maroon}{#1}}}
\newcommand{\note}[1]{{\marginpar{\ttfamily \footnotesize{#1}}}}
\begin{document}
\maketitle
\section{Facts}
	Each time a cell phone connects to a cell tower, it generates a time-stamped record known as cell-site location information (CSLI).
	After the FBI identified the cell phone numbers of robbery suspects, they were able to obtain court orders to obtain the suspects’ cell phone records under the Stored Communications Act.
	They were able to obtain a record of petitioner’s movements over 127 days.
	Petitioner moved to suppress the data, arguing a Fourth Amendment violation.
	The motion was denied and petitioner was convicted.
\section{Procedural History}
	The Sixth Circuit affirmed the District Court’s conviction, holding that Carpenter lacked a reasonable expectation of privacy in location information because that information was shared with his wireless carriers.
\section{Issue}
	\begin{enumerate}
		\item Was the Government’s acquisition of cell-site records a Fourth Amendment search?
		\item Does the search (if any) require a warrant?
	\end{enumerate}
\section{Brief Holding}
	\begin{enumerate}
		\item The Government’s acquisition was a search. When an individual seeks to keep something private, official intrusion into that qualifies as a search.
		This is a complicated case because there are two intersecting interests: location data (protected by \textit{Jones}), but shared with a third party (allowed by \textit{Miller} and \textit{Smith})
		\item This holding is narrow; the warrant is required only in the rare case where the suspect has a legitimate privacy interest in third-party records.
	\end{enumerate}
\section{Majority Opinion}
	\noindent \textsc{Roberts, C.J.} \textit{writing for the Court, joined by} \textsc{JJ. Ginsburg}, \textsc{Breyer}, \textsc{Sotomayor}, \textit{and} \textsc{Kagan} \par
	The purpose of the Fourth Amendment is to safeguard the privacy and security of individuals against arbitrary intrusion. 
	There is no single rubric to determine which expectations of privacy are protected, but we base it off of what was deemed unreasonable when the Amendment was adopted.
	The Court has tried to preserve the privacy that existed when the Amendment was adopted against the encroachment of technology.
	This case sits at the intersection of conflicting jurisprudence on the expectation of privacy over one’s movements and location and the expectation of privacy over information voluntarily shared with a third party.
	\begin{description}
		\item{\textit{United State v. Knotts, 460 U.S. 276 (1983)}}
			A tracking beeper does not constitute a search because a person’s movements on public roads are open to the public, with no reasonable expecation of privacy --- however, the limited use of the beeper is significant, and better surveillance technology may invalidate this.
		\item{\textit{United States v. Jones, 565 U.S. 400 (2012)}}
			FBI agents installed a GPS tracking device. The Court decided the case on the basis of trespass, but five justices concurred that longer term GPS monitoring impinges on expectations of privacy.
		\item{\textit{Smith v. Maryland, 442 U.S. 735 (1979)}}
			A person has no legitimate expectation of privacy in information he voluntarily shares with third parties; a person simply assumes the risk that it will be transmitted to the government.
		\item{\textit{United States v. Miller, 425 U.S. 435 (1976)}}
			...even if the information is revealed on the assumption that it will be used only for a limited purpose
	\end{description}
	The Court does not apply \textit{Smith} and \textit{Miller} because 
	few could have imagined a society in which a phone is carried with its owner and can reveal personal movements.
	People do not surrender all Fourth Amendment protection by venturing into public; there are still things that they can keep private.
	Society does not expect that law enforcement could not monitor every single movement for a very long period.
	Location records hold all the privacies of life.
	Cell phone tracking is cheap and easy, and allows limitless invasion of privacy.
	If the Government didnt’s need a warrant to collect this information, then the government would effectively be surveiling every person all the time.
	Justice Kennedy argues that because CSLI is less precise than GPS data, it is not an invasion of privacy.
	However, an “inference does not insulate a search,” so these things are effectively the same.
	The memories of the logs of corporations are infallible and permanent.
	Applying the third party doctrine would not be a simple application but a huge extension to a new kind of data.
	This is not just a person’s movements on public roads but a complete and exhaustive chronicle of every movement of an individual for years and years.
	Cell phone location information is not truly “shared” as one normally understands the term.
	This decision does not cover CSLI in real time or tower dumps.
	We do not disturb \textit{Smith} and \textit{Miller} or call into question normal surveillance techniques.
	In the absence of a warrant, a search is reasonable only if it falls within a specific exception to the warrant requirement.
	The standards for the SCA are far lower than those for a warrant.
	The order is not a permissible mechanism for obtaining CSLI.
	Alito contends that the warrant requrement does not apply when you go through ome subpoena process.
	However, this is stupid because Alito’s cases don’t compare.


\section{Dissenting Opinion}
	\noindent \textsc{Kennedy, J.} \textit{dissenting, joined by} \textsc{Thomas}, \textit{and} \textsc{Alito, JJ.} \par
	\noindent \textsc{Thomas, J.} \textit{dissenting.}
	\noindent \textsc{Alito, J.} \textit{dissenting.}
	\noindent \textsc{Gorsuch, J.} \textit{dissenting.}
\end{document}
