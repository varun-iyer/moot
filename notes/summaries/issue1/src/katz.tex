\documentclass[paper=letter,fontsize=10pt]{article}
\usepackage{newcent}
\usepackage{lineno}
\usepackage{hyperref}
\usepackage{enumitem}
\usepackage{xcolor}
\title{\textsc{\textit{Katz v. United States} Summary}}
\date{}
\author{\textit{Varun Iyer} \hspace{.5em} $\cdot$ \hspace{.5em} \textit{August 6, 2019}}
\definecolor{maroon}{rgb}{.5,0,0}
\definecolor{navy}{rgb}{0,0,.5}
\newcommand{\pet}[1]{{\color{navy}{#1}}}
\newcommand{\res}[1]{{\color{maroon}{#1}}}
\newcommand{\note}[1]{{\marginpar{\ttfamily \footnotesize{#1}}}}
\begin{document}
\maketitle
\section{Facts}
	Petitioner was convicted of transmitting wagering information across state borders. The government introduced evidence collected through a microphone placed above the telephone box the petitioner was in. Petitioner objected.
\section{Procedural History}
	The District Court found petitioner guilty of eight counts of wagering offenses. The Court of Appeals affirmed his conviction because no physical trespass had occured during the Government’s collection of audio evidence.
\section{Issue}
	The counsel for the petitioner phrased the issue:	
	\begin{enumerate}
		\item Whether a public telephone booth is a constitutionally protected area for the purposes of search and seizure.
		\item Whether physical penetration of a constitutionally protected area is necessary for a Fourth Amendment search and seizure violation.
	\end{enumerate}
	However, the Court rejected this formulation, stating that the Fourth Amendment is not necessarily correctly promoted by the phrase “constitutionally protected area,” and that the Fourth Amendment cannot be translated into a general right to privacy.
\section{Brief Holding}
	\begin{enumerate}
		\item The Government’s eavesdropping violated the petitioner’s justified privacy; the Fourth Amendment protects people, rather than places.
		\item \pet{Although surveillance in this case was narrowly circumscribed, it was not in fact conducted pursuant to the warrant procedure.}
	\end{enumerate}
\section{Majority Opinion}
	\textsc{Stewart, J.}, \textit{joined by} \textsc{Douglas, J.}, \textsc{Warren, J.} \textit{and} \textsc{Fortas, J.} \par
	The Fourth Amendment protects individuals against certain kinds of governmental intrusion. It does not constitute the \textit{general} right of privacy to be let alone.\footnote{See Warren \& Brandeis, \textit{The Right to Privacy}, 4 Harv. L. Rev. 193 (1890)}
	The Fourth Amendment protects people, not places. 
	It does not protect things which people knowingly expose even if those things are within private areas.
	Correspondingly, things which people attempt to keep private even in public areas are protected.
	Even though the phone booth was made of glass, the petitioner’s aim was to protect the uninvited ear. \par
	The Government argues that no physical penetration occured, once the basis of Fourth Amendment inquiry\footnote{\textit{Olmstead v. United States}, 277 U.S. 438, 457, 464, 466; \textit{Olmstead v. United States}, 316 U.S. 129, 134-136}. \pet{However, this basis has been discredited.\footnote{\textit{Warden v. Hayden}, 387 U.S. 294, 304.}} \par
	The fact that the search was narrowly circumscribed is irrelevant. The agents in this case acted with restraint, but this restraint was imposed by the agents themselves, not by a judicial officer. \pet{“[The] Court has never sustained a search upon the sole ground that officers reasonably expected to find evidence of a particular crime and voluntarily confined their activities to the least intrusive means consistent with that end.”}\note{The fact that Chaney and Geesaman’s tracking request was narrowly circumscribed is made entirely irrelevant by this decision.}\footnote{\textit{Ante}, at 357.} Searches conducted without warrants have been held unlawful even if they had probable cause.
\section{Concurring Opinions}
	\textsc{Douglas, J.}\textit{writing for the Court}, \textit{joined by} \textsc{Brennan, J.} \par
		There is no distinction under the Fourth Amendment between types of crimes.
		The fact that a crime is particularly heinous does not grant or remove Fourth Amendment protections. \\
	\noindent \textsc{Harlan, J.} \textit{concurring}\par
		A telephone booth is an area in which a person has a constitutionally protected reasonable expectation of privacy. Electronic and physical intrusion into a place may constitute a Fourth Amendment violation. The invasion of a constitutionally protected area is presumptively unreasaonable absent a search warrant. The majority opinion is not clear enough. The Fourth Amendment protects people, but what protections exist? \par
		There is a twofold requirement:
		\begin{enumerate}
			\item A person has exhibited an actual, subjective expectation of privacy
			\item That expectation is one that society is prepared to recognize as reasonable
		\end{enumerate}
		It is not true that no interception of a conversation can be reasonable without a warrant; warrants are the general rule, to which law enforcement needs may demand specific exceptions.
	\noindent \textsc{White, J.} \textit{concurring}\par
		The warrant procedure should be able to be circumvented for national security reasons by the President or the Attorney General.
\section{Dissenting Opinion}
	\noindent \textsc{Black, J.} \textit{dissenting}\par
		The Fourth Amendment does not mean what the Court has said it means, and it is not the role of the Court to rewrite it according to changing social circumstances or desirability. The language of the Amendment refers explicitly to tangible things; eavesdropping or wiretapping a conversation is not a tangible seizure.
\end{document}
