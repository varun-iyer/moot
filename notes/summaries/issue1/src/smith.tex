\documentclass[paper=letter,fontsize=10pt]{article}
\usepackage{newcent}
\usepackage{lineno}
\usepackage{enumitem}
\usepackage{xcolor}
\title{\textsc{\textit{Smith v. Maryland} Summary}}
\date{}
\author{\textit{Varun Iyer} \hspace{.5em} $\cdot$ \hspace{.5em} \textit{August 6, 2019}}
\definecolor{maroon}{rgb}{.5,0,0}
\definecolor{navy}{rgb}{0,0,.5}
\newcommand{\pet}[1]{{\color{navy}{#1}}}
\newcommand{\res}[1]{{\color{maroon}{#1}}}
\newcommand{\note}[1]{{\marginpar{\footnotesize \texttt{#1}}}}
\begin{document}
\maketitle
\section{Facts}
	A woman was robbed. Following the robbery, she began receiving obscene and threatening phone calls from a person who identified himself as the robber.
	She noted a particular automobile at the scene of the crime and implicated by the calls.
	The police identified the automobile as registered in the name of the petitioner.
	The police requested that the phone company install a pen register at its central offices, and the phone company agreed.
	The pen register recorded that the petitioner had called McDonough. On this evidence, a warrant was issued for petitioner’s home.
	A phone book was found with McDonough’s name and number, and McDonough identified petitioner from a lineup.
	Petitioner moved to supress all fruits of pen register search because of the lack of a warrant. 
\section{Procedural History}
	The trial court admitted evidence and petitioner was convicted and sentenced to six years.
	Maryland Court of Appeals affirmed on the basis that there is no expectation of privacy for numbers dialed into a telephone.
\section{Issue}
	Is the installation and use of a pen register a “search” in the context of the Fourth Amendment?
\section{Brief Holding}
	The installation and use was not a “search,” therefore no warrant is required.
	The application of the Fourth Amendment depends on whether there is is a subjective expectation of privacy exhibited by the individual and that society is prepared to recognize that expectation as reasonable.
	The petitioner did not demonstrate an expectation of prviacy.
	In addition, telephone users in general do not have an expectation of privacy over dialed numbers because they have seen the numbers they dialed recorded for various purposes.\note{This argument may not apply to cell phones --- cell phone users likely do have a reasonable expectation of privacy.}
\section{Majority Opinion}
	\textsc{Blackmun, J.} \textit{delivered the opinion of the Court}, \textit{joined by} \textsc{Burger, C.J.}, \textsc{White, J.}, \textsc{Rehnquist, J.} \textit{and} \textsc{Stevens, JJ.} \par
		The Court explains and remains consistent with the \textit{Katz} analysis.
		The pen register was installed on telephone company property at the telephone company’s central offices. 
		Petitioner cannot claim that his “property” was invaded or that police intruded into a “constitutionally protected area.”
		Pen registers collect a very limited amount of information --- only dialed numbers, not the content of the calls or even whether they are completed.
		All subcribers to cell phones realize that the phone company has facilities to record the numbers they dial, and they see toll calls on their monthly bills.\note{Companies have the \textit{ability} to record just about anything today.  Does this mean we just shouldn’t expect privacy? Companies also establish policies only allowing account holders to access cell phone records, leading people to believe in a reasonable expectation of privacy.}
		At the time, telephone directories state that the companies can help identifying the origin of unwelcome and troublesome calls.
		\res{The petitioner made efforts to hide the content, but not the destination of his calls, so he did not have a reasonable expectation of privacy.}\note{Since Bronner and DeNolf switched burners regularly, they were aware that their phones could and would be tracked and did not demonstrate an expectation of privacy over their cell phones at all.} \par
		Even if the petitioner had a subjective expectation of privacy, it is not reasonable.
		The Court has previously held that information revealed to third parties such as a bank does not have a reasonable expectation of privacy.
		By communicating that information to another person, one runs the risk that it will be conveyed to others or to the government.
		The switching equipment used at the time is just a counterpart of an operator who would personally complete calls for a subscriber.
		Petitioner argues that the distinction between the operator and the switch is that the switch does not remember all calls, only ones it is programmed to remember, so there is an expectation of privacy for local calls.
		However, this distinction is constitutionally irrelevant; the important thing is that the phone company could record dialed numbers and the petitioner was aware of that fact. \par
		For these reasons there was no reasonable expectation of privacy. The judgement of the Court of Appeals is affirmed.
\section{Dissenting Opinion}
	\noindent \textsc{Stewart, J.} \textit{dissenting, joined by} \textsc{Brennan, J.}\par
	There is one, obvious fact the Court ignores --- stating that dialed numbers must go through the telephone company no more than describes the basic nature of phone calls.
	All telephone calls require telephone company property and payment.
	Despite this, the Court has recognized the importance of telephonic communcation and held that the user of a telephone is entitled to assume an expectation of privacy.
	Telephone numbers dialed are contentful.
	Most people have numbers listed in a public directory, but most would be unhappy to have a list of their calls broadcast, revealing intimate details about their life. \\
	\noindent \textsc{Marshall, J.} \textit{dissenting, joined by} \textsc{Brennan, J.}\par
		The Court infers from long-distance listings and cryptic “tracing help” messages in “most” phone books that pen registers are regularly used for recording local calls.
		Even if we assume that individuals expect their numbers to be recorded somewhere, it does not follow that they expect this information to be made public.
		Implicit in the assumption of risk is a notion of choice. In third-party surveillance risk analyses, the defendant has exercised some discretion in deciding who to trust.
		\pet{By contrast in this case, unless someone is prepared to forgo a necessity, they cannot help but accept the risk of surveillance.}\note{Sounds a lot like the case for modern phones.}
		Speaking of risks in situations where there is no choice but to accept risks is useless.
		Making the risk analysis dispositive would \textbf{allow the government to define the scope of Fourth Amendment protections.} 
		\pet{Law enforcement officials, simply by announcing their intent to monitor random samples of first-class mail or phone conversations, could require that the public simply take on a risk in such communications from now on.}\footnote{Amsterdam, Perspectives on the Fourth Amendment, 58 Minn. L. Rev. 349, 384, 407}
		Courts must evaluate the intrinsic character of investigative practices.
		If police could monitor dialed numbers without a warrant they could, for example, connect members of unpopular political orgnizations or confidential sources of jounralists.
		Given the Government’s previous reliance on warrantless surveillance, we cannot insulate pen registars from judicial reviews.
\end{document}
