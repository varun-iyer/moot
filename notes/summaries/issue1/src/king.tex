\documentclass[paper=letter,fontsize=10pt]{article}
\usepackage{newcent}
\usepackage{lineno}
\usepackage{hyperref}
\usepackage[inline]{enumitem}
\usepackage{xcolor}
\title{\textsc{\textit{Kentucky v. King} Summary}}
\date{}
\author{\textit{Varun Iyer} \hspace{.5em} $\cdot$ \hspace{.5em} \textit{August 6, 2019}}
\definecolor{maroon}{rgb}{.5,0,0}
\definecolor{navy}{rgb}{0,0,.5}
\newcommand{\pet}[1]{{\color{navy}{#1}}}
\newcommand{\res}[1]{{\color{maroon}{#1}}}
\newcommand{\note}[1]{{\marginpar{\ttfamily \footnotesize{#1}}}}
\begin{document}
\maketitle
\section{Facts}
	Police officers followed a suspected drug dealer to an apartment.
	They smelled marijuana, knocked loudly, and announced their presence.
	They heard noises they believed were consistent with the destruction of evidence.
	They kicked in the door and found respondent and others.
	They found drugs in plain view during a protective sweep.
	Respondent moved to suppress evidence. 
	Motion was denied, but respondent pled guilty conditional upon the right to appeal the suspension ruling.
\section{Procedural History}
	The Kentucky Court of Appeals affirmed but the Supreme Court of Kentucky reversed.
	The Court assumed that exigent circumstances existed, but invalidated the search because the police should have foreseen that their conduct would prompt occupants to destroy evidence.
\section{Issue}
	Does the exigent circumstances exception to the Fourth Amendment
	warrant requirement still apply when police conduct foreseeably prompted
	a defendant to try to destroy evidence?
\section{Brief Holding}
	\begin{enumerate}
		\item The exigent circumstances rule applies when the police do not create exigent circumstances by engaging or threatening to engage in conduct that violates the Fourth Amendment.
		\begin{enumerate}
			\item Searches inside a home are presumptively unreasonable without a warrant, but this may be overcome by exigent cicumstances including the imminent destruction of evidence.
			\item The lower courts have created an exception for exigency created by the conduct of the police, but differ on the test used to deetermine this.
			\item A warrantless entry is reasonable when the police did not create the exigency by engaging in conduct violating the Fourth Amendment.
			\item Placing bad faith or reasonable belief burdens on the police makes policing really difficult and capricious --- exactly how loud is too loud?, e.g.
		\end{enumerate}
		\item Assuming an exigency existed, the officers’ conduct --- knocking and announcing their presence --- was entirely consistent with the Fourth Amendment.
	\end{enumerate}
\section{Majority Opinion}
	\textsc{Alito, J.}, \textit{delivered the opinion of the Court, joined by} \textsc{Roberts, C.J.}, \textsc{Scalia, Kennedy, Thomas, Breyer, Sotomayor,} and \textsc{Kagan, JJ.}. \par
		The Fourth Amendment imposes two requirements. 
		\begin{enumerate*} \item Searches and seizures must be reasonable. \item Warrants may not be issued without probably cause and a particular scope of search. \end{enumerate*}
		A well recognized exception is exigent circumstances.
		These include emergency aid, hot pursuit, and the imminent destruction of evidence.
		Lower courts have stated that the police must be responding to an unanticipated exigency rather than creating the exigency themselves.
		However, this creates an issue with destruction of evidence.
		Evidence is almost always destroyed specifically because of a fear of law enforcemennt, so the police in a way always create the exigency.
		Lower courts have devised many different kinds of tests to circumvent this issue.
		If police conduct is consistent with the Fourth Amendment, then a warrantless search under exigent circmustances is permissible.
		This doctrine is consistent with other warrantless search requirements.
		If an officer sees evidence in plain view, they may enter without a warrant in order to seize it.
		The Fourth Amendment requires only that the seizure and the steps preceding the seizure be lawful, and this doctrine is consistent with that requirement.
		Most other tests are too nebulous and difficult to apply. 
		Individuals retain the right to simply not answer the door or not allow officers in; individuals who choose to destroy evidence have only themselves to blame.
\section{Dissenting Opinion}
	\noindent \textsc{Ginsburg, J.} \textit{dissenting}\par
		The Fourth Amendment does not mean what the Court has said it means, and it is not the role of the Court to rewrite it according to changing social circumstances or desirability. The language of the Amendment refers explicitly to tangible things; eavesdropping or wiretapping a conversation is not a tangible seizure.
\end{document}
