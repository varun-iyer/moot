\documentclass[paper=letter,fontsize=10pt]{article}
\usepackage{newcent}
\usepackage{lineno}
\usepackage{hyperref}
\usepackage{enumitem}
\usepackage{xcolor}
\title{\textsc{\textit{Kyllo v. United States} Summary}}
\date{}
\author{\textit{Varun Iyer} \hspace{.5em} $\cdot$ \hspace{.5em} \textit{August 6, 2019}}
\definecolor{maroon}{rgb}{.5,0,0}
\definecolor{navy}{rgb}{0,0,.5}
\newcommand{\pet}[1]{{\color{navy}{#1}}}
\newcommand{\res}[1]{{\color{maroon}{#1}}}
\newcommand{\note}[1]{{\marginpar{\ttfamily \footnotesize{#1}}}}
\begin{document}
\maketitle
\section{Facts}
	Agents of the Department of the Interior began to suspect that marijuana was being grown inside petitioner’s home. In order to determine this, agents used a thermal imager to scan the home and found that heat radiation was consistent with high-intensity lamps used to grow marijuana. 
	Based on tips from informants, utility bills, and thermal imaging, a warrant was issued to search the home and petitioner was convicted for manufacturing marijuana.
\section{Procedural History}
	Petitioner unsuccessfully moved to suppress evidence seized from the heat camera and following searches. 
	The Ninth Circuit remanded for a hearing regarding the intrusiveness of thermal hearing, and the District Court found that the imaging was non-intrusive.
	After the evidentiary hearing, the Ninth Circuit affirmed the conviction because the imager did not expose intimate details.
\section{Issue}
	Is the use of a thermal imaging camera a ‘search’ in the context of the Fourth Amendment?
\section{Brief Holding}
	\begin{enumerate}
		\item A search occurs when a reasonable expectation of privacy exists.
		\item Sense-enhancing technology that reveals information about a home’s interior intrudes into a private area.
		\item Based on this, the information obtained by the thermal imager was the product of a search.
	\end{enumerate}
	The ruling of the Circuit is reversed and remanded to the District Court.
\section{Majority Opinion}
	\textsc{Scalia, J.}, \textit{delivered the opinion of the Court, joined by} \textsc{Souter, J.}, \textsc{Thomas, J., Ginsburg, J.} \textit{and} \textsc{Breyer, J.} \par
		The core of the Fourth Amendment is to be free from unreasonable intrustion.
		With a few exceptions, a warrantless search of a home is unreasonable and therefore unconstitutional.
		The question is whether the use of this device with no physical trespass constitutes a search or not.
		\pet{The violation of Fourth Amendment rights has been decoupled from the trespassory violation of his property.}\note{Promotes the dismissal of \textit{Jones} test.}
		Technology has clearly diminished personal privacy. 
		The question confronted is what limits there are upon this shrinkage.
		The \textit{Katz} test is problematic and possibly circular.
		Howeber, this is a clear-cut case: the interior of a house is certainly private.
		The expetation of privacy exists and is reasonable.
		Sense-enhancing technology that reveals information otherwise unobtainable but by physical intrusion constitutes a search. \par
		The dissent argues that the device only captures things emanating from the house, therefore unprotected.
		However, in \textit{Katz}, the Government argued that the only thing being captured was emanating wound waves, and this was soundly rejected.
		Allowing emanations to be unprotected leaves individuals at the mercy of advancing technology.
		In addition, the Fourth Amendment’s protection has never been based on the quantity and quality of the information captured.
		In a home, all details are intimate details.\footnote{\textit{Arizona v. Hicks}, 480 U.S. 321 (1987), holding that turning over a phonograph turntable was a ‘search.’}
		No important information need be found for a violation of privacy to occur.
		When the Government uses a device that is not in general public use, to explore detals of the home that would previously have been unknowable without physical intrusion, the surveillance is a “search,” and is presumptively unreasonable.
\section{Dissenting Opinion}
	\noindent \textsc{Stevens, J.} \textit{dissenting, joined by } \textsc{Rehnquist, C.J., O’Connor, J.,} \textit{and} \textsc{Kennedy, J.}\par
		There is a constitutionally significant distinction between “through-the-wall surveillance” which gives the observer direct acces to information in a private area, and “off-the-wall surveillance,” which uses indirect inferences made from observations of the exterior of the home.
		The infrared camera simply measured heat emanating from the exterior surfaces of petitioner’s home.
		Passive detection of this emanation does not intrude upon any part of the interior of the house.
		A passerby might notice that the building was emanating heat, or that snow was melting unusually accross its surfaces.
		Heat waves, aromas, and other emissions enter the public domain if and when they leave a building. 
		A subjective expectation that they would remain private is implausible and unreasonable.
\end{document}
