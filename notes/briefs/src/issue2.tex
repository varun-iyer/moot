\documentclass[paper=letter,fontsize=10pt]{article}
\usepackage{newcent}
\usepackage{lineno}
\usepackage{enumitem}
\title{\textsc{Issue Two Case Briefs}}
\newcommand{\case}[3]{\noindent {\large\textbf{\textit{#1}, #2 (#3)}} \par}
\date{}
\author{\textit{Varun Iyer} \hspace{.5em} $\cdot$ \hspace{.5em} \textit{July 14, 2019}}
\begin{document}
\maketitle
\begin{center} 
	\textit{Was Bobby Bronner’s Sixth Amendment right to confront his accuser vioated by the introduction of Andy Sommerville’s hearsay decarations?}
\end{center}
\case{Maryland v. Craig}{497 U.S. 836}{1990}
	\begin{description}[align=right]
		\item[Facts] \noindent
			Sandra Ann Craig, the operator of a kindergarten and pre-school facility, was accused of sexually abusing a six-year old child.
			The trial court allowed the alleged child victim to testify via one-way CCTV. 
			The child testified outside the courtroom, but Mrs. Craig could make objects through electronic communication with her lawyer.
			The judge and jury also viewed the testimony in the courtroom.
			This was done to avoid the possiblity of serious emotional distress for the child witness.
		\item[Issue] \noindent
			Did the CCTV testimony violate the Confrontation Clause of the Sixth Amendment?
		\item[Holding] \noindent 
			No. Justice O’Connor wrote for the majority. The Court held that the Confrontation Clause was not absolute.
			In certain narrow circumstances, competing interests, if closely examined, may warrant having no confrontation at trial.
			The State’s interest in protecting physical and psychological well-being of children could be important enough to outweigh defendants’ rights to face their accusers in court.
			In addition, the child witness was cross-examined by the defendant’s attorney and her general demeanor was visible in the courtroom, giving the defendant a constitutionally sufficient opportunity to test the credibility an substance of her tetimony.
		\item[Dissent] \noindent
			Justice Scalia wrote the dissent, saying that he was “persuaded... that the Maryland procedure is virtually constitutional. Since it is not, however, actually constitutional, I would affirm the judgement of the Maryland Court of Appeals...”
		\item[Rules] \noindent
			The Confrontation Clause is not absolute, allowing narrowly tailored exceptions for compelling government interests.
	\end{description}
\case{United States v. Gigante}{166 F.3d 75}{2d Cir. 1999}
	\begin{description}[align=right]
		\item[Facts] \noindent
			Vincent Gigante was alleged to be the ultimate authority behind many murders and racketeering the window industry. During the trial, the judge allowed Peter Savino, a crucial witness, to testify via two-way CCTV due to a fatal illness which required medical supervision at a remote location.
		\item[Issue] \noindent
			Does the Sixth Amendment’s Confrontation Clause require a face-to-face confrontation?
		\item[Holding] \noindent 
			No. The central concern of the Confrontation Clause is to ensure the reliabilit of evidence against a criminal defendant by subjecting it to rigorous testing, including testimony under oath, cross-examination, the observation of demeanor, and reduced risk of wrongful implication.
			Because CCTV can preserve all of these characteristics, it can satisfy constitutional confrontation requirements.
			F.R.Crim.P. 15(e) allows a trial court to order this kind of tesimony when a witness “is unable to be present or to testify because of... physical or mentall illness or infirmity.”
		\item[Rules] \noindent
			The Confrontation Clause does not require a face-to-face confrontation.
	\end{description}
\case{Crawford v. Washington}{541 U.S. 36}{2004}
	\begin{description}[align=right]
		\item[Facts] \noindent
			The petitioner stabbed a man who allegedly tried to rape his wife. He and his wife were arrested and interrogated, and his wife’s statement cast doubt on her husband’s assertion of self-defense.
			At the trial, his wife’s tape was used, but she did not testify due to Washington’s marital privilege rule.
			The petitioner objected, but the Washington Court of Appeals upheld the conviction after a nine-part test.
		\item[Issue] \noindent
			Whether the State’s use of Sylvia’s statement violated the Confrontation Clause.
		\item[Holding] \noindent 
			Yes. After examining the history of the Clause, the Supreme Court arrived at two conclusions: the principal evil at which the Clause was directed was civil-law mode of criminal provedure and its use of ex-parte examinations as evidence against the accused.
			Second, that the Framers would not have allowed admission of testimony unless the witness was unavailable to testify, and the defendant had a prior opportunity for cross-examination, neither of which is true in this case.
		\item[Dissent] \noindent
			Justice Rehnquist and O’Connor disagreed.
		\item[Rules] \noindent
			The Sixth Amendment requires the opportunity for cross examination.
	\end{description}
\case{Davis v. Washington}{547 U.S. 813}{2006}
	\begin{description}[align=right]
		\item[Facts] \noindent
			This case combined two state Supreme Court Cases, one from Indiana and one from Washington.
			In Washington, testimony by a 911 operator about a caller identifying the plaintiff as her assailant was ruled to be inadmissible hearsay.
			In Indiana, the plaintiff argued that a police officer‘s testimony about statements made by the alleged victim at the crime scene was inadmissible hearsay.
		\item[Issue] \noindent
			Are statements made to law enforcement at the crime scene (or on the telephone) “testimonial” and subject to the requirement of the Confrontation Clause?
		\item[Holding] \noindent 
			A statement made to 911 was not testimonial because it is not designed to establish or prove facts but to describe circumstances about police assistance.
			The caller spoke about events as they were actually occurring rather than describing the past.
			In the case of the police, the victims were made in response to an officer’s question to establish facts about the past.
		\item[Rules] \noindent
			Statements intended to establish or prove facts are testimonial.
	\end{description}
\case{Ohio v. Clark}{135 S.Ct. 2173}{2015}
	\begin{description}[align=right]
		\item[Facts] \noindent
			Respondent Darius Clark sent his girlfriend away to engage in prostitution while he cared for her 3-year-old son L.P. and 18-month-old daughter A.T. When L.P.’s preschool teachers noticed marks on his body, he identified Clark as his abuser. Clark was subsequently tried on multiple counts related to the abuse of both children. At the trial, L.P.’s statements to his teachers was introduced as evidence, but L.P. did not testify.
		\item[Issue] \noindent
			Does the introduction of a child’s statements at trial violate the Sixth Amendment’s Confrontation Clause even if the child does not physically testify?
		\item[Holding] \noindent 
			No. The child’s statements did not violate the Confrontation Clause because they were not testimonial since they were not made with the primary purpose of creating evidence for the respondent’s prosecution --- they occured in the context of an ongoing emergency.
			The child’s teachers asked questions aimed at identifying and ending a threat.
			They did not inform the child that his answers would be used against his abuser.
			In the informal and spontaneous conversation between the child and his teachers, the former never hinted that he intended his statements to be used by the police or prosecutors. The child’s age further confirms the statements were not testimonial because statements with young children rarely are.
		\item[Rules] \noindent
			In the context of the Confrontation Clause, the informality of the situation and interrogation is a relevant factor.
	\end{description}
\case{United States v. Yates}{438 F.3d 1307}{11th Cir. 2006}
	\begin{description}[align=right]
		\item[Facts] \noindent
			Anton Pusztai and Anita yates were tried for mail fraud, conspiracy, and prescription drug related offenses. 
			The Government moved to allow the introduction of testimony from two witnesses in Australia, an alleged payment processor and a doctor.
			Both witnesses were unwilling to travel to the United States and are beyond the government’s subpoena powers.
			They were sworn in by a deputy clerk and understood that their testimony was under oath. 
			Each defendants’ attorney cross-examined both witnesses, but the jury found them guilty on all counts.
		\item[Issue] \noindent
			Does witness testimony over a two-way television video conference violate the defendants’ Sixth Amendment confrontation right?
		\item[Holding] \noindent 
			Yes, it was a violation of the confrontation right.
			In this situation, furtherance of the public policy of expeditious resolution outweighs the defendant’s 6th Amendment confrontation right. If the Court did allow this, the Government would find it convenient to present video testimony in any case, and the Government’s public policy argument does not distinguish between these types of cases.
		\item[Rules] \noindent
			Convenience without exceptional circumstances does not outweigh a Sixth Amendment confrontation right.
	\end{description}
\case{State v. Henderson}{160 P.3d 776}{Kan. 2007}
	\begin{description}[align=right]
		\item[Facts] \noindent
			Elroy D. Henderson was accused of aggravated indecent liberties with a minor.
			His former girlfriend noticed that her daughter had unusual vaginal discharges and upon medical examination, found that she had contracted gonorrhea and had a urinary tract infection.
			When asked who would have unsupervised access to her child, she gave the name of her ex-boyfriend. 
			During the trial, the jury watched a video recording of a detective interviewing F.J.I. (the daughter), who identified various body parts and asserted that 'Tae' (Elroy) had touched her with his 'ding ding'. 
			The jury found Elroy guilty, but the Court of Appeals reversed, agreeing with the defendant that the video testimony was a violation of his confrontation right.
		\item[Issue] \noindent
			Was F.J.I.’s interview testimonial and therefore subject to Confrontation Clause guarantees?
		\item[Holding] \noindent 
			Yes, it was. Because the primary purpose of the investigators (adopting a \textit{Davis} analysis with focus on the questioner) was not to further the welfare of the child or aid in an ongoing emergency, the statements were testimonial. 
		\item[Concurrence] \noindent
		\item[Rules] \noindent
			If the primary purpose of the investigators is to collect evidence for prosecution, then the statements collected during the course of an interview are testimonial.
	\end{description}
\case{State v. Contreras}{979 So. 2d 896}{Fla. 2008}
	\begin{description}[align=right]
		\item[Facts] \noindent
			Rodolfo Contreras was convicted of molesting his nine-year old daughter on the basis of evidence which included a videotaped interview with the daughter.
			His daughter did not testify in court based on the testimony of a psychologist who stated that she would undergo severe emotional harm.
			Contreras moved for a judgement of acquittal because the interview was taken without an opportunity for cross examination.
			Hoewver, the defense had conducted a discovery deposition\footnote{The sworn testimony of a witness taken in the pre-trial discovery phase; attorneys for both sides have an opportunity to question} of his daughter and did not introduce its results to trial. 
			The trial judge denied the motion under the weight of the evidence and the satisfaction of the confrontation right through the deposition, but the appeals court reversed the decision under \textit{Crawford}’s physical unavailability requirement.
		\item[Issue] \noindent
			Can a discovery deposition satisfy the \textit{Crawford} requirement of a prior opportunity for cross examination of a witness?
		\item[Holding] \noindent 
			Because the interview’s purpose was to establish prior facts televant to prosecution and was conducted in police presence and aided by police questions, under the \textit{Davis}  and \textit{Crawford} tests, the videotaped evidence was testimonial.
			However, the child witness is unavailable for the purposes of \textit{Crawford} because of the psychological evaluation indicating likelihood of severe mental and emotional harm and her protected status as a minor.
			The discovery deposition was not an adequate substitute for cross-examination because the defendant was not present, did not know it would be his only opportunity, and cannot efficiently cross examine matters he is learning of for the first time.
			Therefore, the admission of the child’s statements violated Contreras’ Sixth Amendment right.
		\item[Rules] \noindent
			The Confrontation Right does not need to be fulfilled under \textit{Crawford} if:
			\begin{enumerate}
				\item The witness’ statements were not testimonial
				\item The witness is unavailable
				\item The defendant had some prior opportunity for cross-examination or equivalent
			\end{enumerate}
	\end{description}
\case{Seely v. State}{282 S.W. 3d 778}{Ark. 2008}
	\begin{description}[align=right]
		\item[Facts] \noindent
		\item[Issue] \noindent
		\item[Holding] \noindent 
		\item[Concurrence] \noindent
		\item[Dissent] \noindent
		\item[Rules] \noindent
	\end{description}
\end{document}
