\documentclass[paper=letter,fontsize=10pt]{article}
\usepackage{newcent}
\usepackage{lineno}
\usepackage{enumitem}
\title{\textsc{Issue One Case Briefs}}
\newcommand{\case}[3]{\noindent {\large\textbf{\textit{#1}, #2 (#3)}} \par}
\date{}
\author{\textit{Varun Iyer} \hspace{.5em} $\cdot$ \hspace{.5em} \textit{July 13, 2019}}
\begin{document}
\maketitle
\begin{center} 
	\textit{Were Bobby Bronner’s Fourth Amendment rights were violated by
	obtaining his cellphone records without a warrant?}
\end{center}
\case{Katz v. United States}{389 U.S. 347}{1967}
	\begin{description}[align=right]
		\item[Facts] \noindent
			The petitioner, Katz, trasmitted wagering information over telephone lines in violation of federal law. Without warrant, the FBI placed a listening device in a telephone booth used by the petitioner and used the recorded conversations as evidence against him during his trial.
		\item[Issue] \noindent
			Whether the Fourth Amendment of the Constitution protects secret, unwarranted recordings of conversations in a phone booth as evidence against a defendant?
		\item[Holding] \noindent 
			Justice Stewart wrote for the majority that the Fourth Amendment “protects persons and not places” from unreasonable intrusion.
			The petitioner did not hide himself from the public view but did seek to keep the content of his conversation private. 
			A person who enters a telephone booth may expect Fourth Amendment protections as he assumes that the words he utters will not be broadcast to the world. 
			The government’s actions constituted a search, and without a warrant predicated on probable case, all evidence resulting from the search is inadmissible.
		\item[Concurrence] \noindent
			Justice Harlan’s concurrence has become the primary font of precendent from this case. 
			He created a twofold requirement for the protection afforded under the Fourth Amendment. 
			First, that a person has exhibited an actual expectation of privacy, and second, that society is prepared to accept that expectation as reasonable.
		\item[Dissent] \noindent
			Justice Black dissented, writing that eavesdropping was an ancient practice of which the Framers were certainly aware.
			There is no language in the Constitution which prohibits eavesdropping; the Framers very clearly and specifically wrote about searches and seizures of things that can be searched and seized, not conversations.
		\item[Rules] \noindent
			The often-cited rule from this case is the \textit{reasonable expectation of privacy} which gives evidence Fourth Amendment protections.
	\end{description}
\case{Smith v. Maryland}{442 U.S. 735}{1979}
	\begin{description}[align=right]
		\item[Facts] \noindent
			A woman was robbed and received threatening phone calls from the robber. Without obtaining a warrant, the police requested that the telephone company record numbers dialed from the robbers house using a pen register.
		\item[Issue] \noindent
			Was the installation and use of a pen register a “search”?
		\item[Holding] \noindent
			A pen register is not a search because the “petitioner voluntarily conveyed numerical information to the telephone company”; no warrant is required. 
			The defendant knew the numbers he dialed would go to the phone company to connect his call, so there is no reasonable expectation of privacy. Dialing the numbers to automatic equipment is the same as telling the numbers to a human operator. Pen registers and other metadata collection is completely outside Constitutional protection.
		\item[Dissent] \noindent
			Justice Stewart wrote for himself and Justice Brennan. A list reveals the identities of the persons and places, which reveals intimate details. Just because telephone companies know the calls internally doesn’t mean that a person doesn’t reasonable expect that this information wouldn’t be disclosed to the public or to the government.
		\item[Rules] \noindent
			This case created the \textit{Third Party Doctrine}: there is no reasonable expectation of privacy for information voluntarily given to third parties.
	\end{description}
\case{Kyllo v. United States}{533 U.S. 27}{2001}
	\begin{description}[align=right]
		\item[Facts] \noindent
			The police obtained evidence of a cannabis growing operation inside Kyllo’s home by using a thermal imaging device. The police used the device to gather evidence to support issuance of a search warrant for the home.
		\item[Issue] \noindent
			Does the use of a device by the government to obtain evidence from a constitutionally protected area without physical intrusion constitute a Fourth Amendment search?
		\item[Holding] \noindent
			Justice Scalia wrote for the 5-4 majority. When police obtain information using a device which is not publicly used, they are conducting a search. As technology becomes increasingly sophisticated, the right to privacy could become meaningless if there is no limitation upon its use. There is no meaningful distinction between surveillance planted inside the house and surveillance by technology from without.
		\item[Dissent] \noindent
			Justice Stevens argued that thermal imaging did not constitute a search because any person could detect heat emissions by feeling some areas around the house or noticing the snow melting more or less quickly around certain areas of the house. Also, it’s a stretch to bring something as intangible as heat into the private sphere.
		\item[Rules] \noindent
			The use of a device by the government which is not generally used by the public is presumptively unreasonable under the Fourth Amendment.
	\end{description}
\case{United States v. Jones}{565 U.S. 400}{2012}
	\begin{description}[align=right]
		\item[Facts] \noindent
			Jones (respondent) was the owner and operator of a nightclub who was suspected of drug trafficking. 
			The government obtained a warrant to authorize the use of a GPS tracking device on a Jeep registered to Jones’ wife (of which Jones was the exclusive driver), but did not comply with the warrant’s deadline. 
			The government tracked the position of the Jeep over a 28-day period, indicting Jones with charges including conspiracy to distribute cocaine.
		\item[Issue] \noindent
			Does the attachment of a GPS tracking device to a vehicle and subsequent use of that device to monitor the vehicle’s movements on public streets constitute a search or seizure within the meaning of the Fourth Amendment?
		\item[Holding] \noindent
			Justice Scalia wrote for the majority. The government’s installation and use of a GPS device constituted a search.
			Physical intrusion on Jones’ car would clearly be a search, and physical intrusion was required in order to plant the device. In addition, Jones had a reasonable expectation of privacy over his movements, and would not believe that any person or organization could closely track his movements over such a long period.
		\item[Concurrence] \noindent
			Alito wrote for Ginsburg, Breyer, and Kagan. Not every trespass is a search; the case should only be analyzed under Katz. Short term monitoring on public streets may have been acceptable, but longer term monitoring was a breach of a reasonable expectation. \par
			Sotomayor wrote independently, noting that even short-term monitoring may violate a reasonable expectation of privacy because of the unique nature of GPS surveillance.
		\item[Rules] \noindent
			The case was decided based on the black letter of the Fourth --- physical trespass, supplemented by the \textit{Katz} test.
	\end{description}
\case{Kentucky v. King}{563 U.S. 452}{2011}
	\begin{description}[align=right]
		\item[Facts] \noindent
			Police officers smelled marijuana outside an apartment door while pursuing a drug dealer. 
			They knocked on the apartment door. Believing that they heard noises sounding like the destruction of evidence, the officers kicked down the door and found King and others in the apartment with drugs in plain view.
			Drugs and paraphernalia were found in a subsequent search of the apartment.
		\item[Issue] \noindent
			Does the exigent circumstances exception to the Fourth Amendment warrant requirement still apply when police conduct foreseeably prompted a defendant to try to destroy evidence?
		\item[Holding] \noindent
			Justice Alito wrote for the majority. Because knocking and announcing their presence is police conduct that is consistent with the Fourth Amendment, the police did not create the exigency.
		\item[Dissent] \noindent
			Justice Ginsburg dissented. Because of the Court’s decision, officers may knock, listen, and then break down someone’s door without running afoul of the Fourth Amendment.
			Exigent circumstances must exist when the police arrive on the scene, not after their arrival and prompted by their own conduct.
		\item[Rules] \noindent
			Exigent circumstances do not apply if the police create the exigent circumstance; however, conduct consistent with the Fourth Amendment is not the creation of an exigent circumstance.
	\end{description}
\case{Missouri v. McNeely}{569 U.S. 141}{2013}
	\begin{description}[align=right]
		\item[Facts] \noindent
			McNeely was stopped after a highway patrol officer observed him exceed the speed limit and cross the centerline. 
			The officer noticed signs of intoxication. McNeely failed field-sobriety tests.
			McNeely refused to blow into a breathalyzer and stated that he would refuse on at the station. 
			The officer drove McNeely to the hospital in order to draw blood to measure McNeely’s BAC.
			McNeely was charged with a DUI, but attempted to suppress the results of the blood test because it was an unconstitutionally unreasonable search and seizure.
		\item[Issue] \noindent
			Does the Fourth Amendment prevent the taking of a warrantless blood sample under exigent circumstances?
		\item[Holding] \noindent
			Sotomayor wrote for the 5-4 plurality. The Fourth Amendment’s protection against warrantless searches applies to blood alcohol tests unless there are exigent circumstances.
			Each case must be considered based on its individual facts, and there are some circumstances in which natural dissipation of alcohol would be an exigent circumstances, but there is no reason for a categorical rule.
			The Fourth Amendment’s protection against bodily intrusions outweighs the state’s interest in gaining evidence quickly.
		\item[Concurrence] \noindent
			Justice Kennedy partially concurred. The Fourth Amendment does not provide the basis for a categorical rule, and the warrant requirement cannot be ignored in drunk driving arrests. \par
			Chief Justice Roberts concurred in part and dissented in part, along with Justices Breyer and Alito. There must be a categorical rule in order to provide guidance to law enforcement officials.
			Exigent circumstances exist and justify a blood test if the officer believes there is not sufficient time to botain a warrant before evidence is lost through the metabolic process.
			If there is time to secure a warrant, the officer must do so.
		\item[Dissent] \noindent
			Justice Thomas dissented, arguing that the body’s metabolization of alcohol constituted destruction of evidence, creating an exigent circumstance.
		\item[Rules] \noindent
			The Court avoided creating a categorical rule for this issue.
	\end{description}
\case{Riley v. California}{573 U.S. 373}{2014}
	\begin{description}[align=right]
		\item[Facts] \noindent
			Riley was pulled over for a traffic violation and was arrested for weapons-related charges. Riley was searched after his arrest and his cell phone was seized. Riley was convicted after a trial where evidence from his phone was introduced in a shooting related charge.
			Wurie, another defendant, was searched in his home based on evidence found on his phone without warrant. The appeals court ruled that the evidence found in the home could not be submitted as it was fruit of the poisoned tree.
		\item[Issue] \noindent
			May the government conduct a warrantless search of the contents of a cell phone seized after an arrest when no exigent circumstances exist?	
		\item[Holding] \noindent
			No. The exception to a search incident to arrest is permitted for officer safety and to prevent destruction of evidence.
			No safety risk exists in a cell phone beyond a preliminary physical search to ensure that no weapon is present.
			Once officers have secured a cell phone, there is little risk of destruction of stored evidence. 	
			The search of cell phone data is a major invasion of privacy due to the quality and quantity of cell phone information.
			The government may not conduct a search without a warrant or exigent circumstances.
		\item[Concurrence] \noindent
			Justice Alito concurred. The history of the search-incident-to-arrest exception is substantially based on provative evidence and not destruction of evidence. Regardless, the majority correctly holds the rules of a physical search do not apply to cell phone data.
		\item[Rules] \noindent
			The government may not conduct a warrantless search of a cell phone incident to arrest in the absence of exigent circumstances.
	\end{description}
\case{Carpenter v. United States}{138 S.Ct. 2206}{2018}
	\begin{description}[align=right]
		\item[Facts] \noindent
			Each time a phone connects to a cell site, it generates a time-stamped record known as cell-site location information (CSLI).
			After the FBI identified the cell phone numbers of several robbery suspects, prosecutors were granted court orders to obtain suspects’ cell phone  records under the Stored Communications Act.
			The petitioner moved to suppress the data, arguing that the seizure of CSLI without a warrant supported by probable cause violated the Fourth Amendment.
		\item[Issue] \noindent
			Did the government violate petitioner’s right to privacy when they accessed his CSLI without a warrant?
		\item[Holding] \noindent
			Justice Roberts wrote for the majority, joined by Ginsburg, Breyer, Sotomayor, and Kagan. Yes. Government’s acquisition of historical CSLI was a search under the Fourth Amendment.
			When the government accessed the defendant’s CSLI, it invaded his reasonable expectation of privacy in the whole of his physical movements, and the fact that the government obtained the information from a third party did not overcome the defendant’s Fourth Amendment protections.
			A court order under the Stored Communications Act was not a permissible mechanism because the evidence required fell well short of probable cause. 
			In the exception of exigent circumstances, a warrant was necessary to obtain CSLI.
		\item[Dissent] \noindent
			Justice Kennedy dissented, arguing that the Court’s new formulation puts reasonable and congressionally authorized criminal investigation at risk in serious cases. It places undue restrictions on the lawful and necessary enforcement powers of the state and federal governments.\par
			Justice Alito dissented, arguing that the Court’s concern about new technology fractured the Third Party Doctrine and the \textit{Katz} test, guaranteeing a blizzard of litigation and threatening valuable investigative practices.\par
			Justice Gorsuch dissented, agreeng with the majority’s decision, but disagreeing with their reasoning. He disagreed that the Fourth Amendment provides a right to a “reasonable expectation of privacy,” instead arguing that CSLI records are the property of cell phone owners, and under the Fourth Amendment, law enforcement agencies cannot search property without a warrant. Gorsuch argues that \textit{Katz}, \textit{Smith}, and \textit{Miller} are inconsistend and should be overturned.
		\item[Rules] \noindent
			Accessing CSLI records constitutes a search and requires a warrant in the absence of exigent circumstances.
	\end{description}

\end{document}
